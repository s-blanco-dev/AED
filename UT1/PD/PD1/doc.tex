\documentclass[12pt,letterpaper, onecolumn]{exam}
\usepackage{amsmath}
\usepackage[T1]{fontenc}
\usepackage{amssymb}
\usepackage[a4paper, total={6.5in, 10in}]{geometry}
\usepackage[spanish]{babel}
% \chead{\hline} % Un-comment to draw line below header
\thispagestyle{empty}   %For removing header/footer from page 1

\begin{document}

\begingroup  
    \centering
    \LARGE Algoritmos y Estructuras de Datos\\
    \large Unidad Temática 1\\
    \large Práctico Individual 1\\[0.5em]
    \normalsize \today\\[0.5em]
    \normalsize Santiago Blanco Canaparro\par
    \normalsize Profesor: Sebastián Torres\par
    \normalsize Grupo I2M2\par
\endgroup
\rule{\textwidth}{0.4pt}
\pointsdroppedatright   %Self-explanatory
\printanswers
\renewcommand{\solutiontitle}{\noindent\textbf{Respuesta:}\enspace}   %Replace "Ans:" with starting keyword in solution box

\begin{questions}

    \question[] Ejercicio 1\droppoints
    
    \begin{solution}
      \\
      No yo pac.\textbackslash n\\
      Vos zacata pac.\textbackslash n\\
      Yo pac.\textbackslash n
    \end{solution}
    
    \question[] Ejercio 2\\
    (a.) ¿Cuál es la primera sentencia que se ejecuta?
    
       
    \begin{solution}
        sipo ("traqueteo", 13);
    \end{solution}

    (b.) ¿Cuál es la primera sentencia que se ejecuta?

    \begin{solution}
      \begin{verbatim}
        public class Zumbido {
public static void desconcertar (String dirigible) {
6. System.out.println (dirigible);
7. sipo ("ping", -5);
}
public static void sipo (String membrillo, int flag) {
2, 8. if (flag < 0) {
9. System.out.println (membrillo + " sup");
3. } else {
4. System.out.println ("ik");
5. desconcertar (membrillo);
10. System.out.println ("muaa-ja-ja-ja");
}
}
public static void main (String[] args) {
1. sipo ("traqueteo", 13);
}
}
      \end{verbatim}
    \end{solution}    

    \pagebreak %Not necessary
    
    \question[] Ejercicio 3\droppoints
    
    \begin{solution}
            \begin{verbatim}

public class Multisuma {
    public static double multisuma(double a, double b, double c) {
        return a*b + c;
    }

    public static void main(String[] args) {
        double resultado = multisuma(1.0,2.0,3.0);
        System.out.println(resultado);
    }
}
            \end{verbatim}
    \end{solution}
    
\end{questions}
\end{document}
