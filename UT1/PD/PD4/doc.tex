\documentclass[12pt,letterpaper, onecolumn]{exam}
\usepackage{amsmath}
\usepackage[T1]{fontenc}
\usepackage{amssymb}
\usepackage[a4paper, total={6.5in, 10in}]{geometry}
\usepackage[spanish]{babel}
% \chead{\hline} % Un-comment to draw line below header
\thispagestyle{empty}   %For removing header/footer from page 1

\begin{document}

\begingroup  
    \centering
    \LARGE Algoritmos y Estructuras de Datos\\
    \large Unidad Temática 1\\
    \large Práctico Individual 4\\[0.5em]
    \normalsize \today\\[0.5em]
    \normalsize Santiago Blanco Canaparro\par
    \normalsize Profesor: Sebastián Torres\par
    \normalsize Grupo I2M2\par
\endgroup
\rule{\textwidth}{0.4pt}
\pointsdroppedatright   %Self-explanatory
\printanswers
\renewcommand{\solutiontitle}{\noindent\textbf{Respuesta:}\enspace}   %Replace "Ans:" with starting keyword in solution box

\begin{questions}

    \question[] Ejercicio 1\droppoints
    
(a.) ¿Cuáles son las variables de clase? \\
    (b.) ¿Cuáles son las variables de instancia?
    \begin{solution}
      \\
      \textbf{public static int x = 7;} es la variable de clase porque la palabra \textbf{static} establece que sea compartida por todas las instancias.\\
      \textbf{public int y = 3;} es la variable de instancia.
    \end{solution}

    La salida producida por el código:
    \begin{solution}
    \begin{verbatim}
    a.y = 5
    b.y = 6
    a.x = 2
    b.x = 2
    IdentifyMyParts.x = 2
    \end{verbatim}
    \end{solution}
    
    \question[] Ejercio 2\\
    (a.) Indica lo que está mal en el siguiente programa:
    \begin{verbatim}
public class SomethingIsWrong {
    public static void main(String[] args) {
      Rectangle myRect;
      myRect.width = 40;
      myRect.height = 50;
      System.out.println("myRect's area is " + myRect.area());
    }
}
    \end{verbatim}
    
       
    \begin{solution}
      El problema con el programa es que el objeto \textbf{myRect} nunca se inicializa por lo que \textbf{myRect} apunta a null. En ejecución genera una excepción del tipo \textbf{NullPointerException}.
      La solución:
      \begin{verbatim}
      Rectangle myRect = new Rectangle();
      \end{verbatim}

    \end{solution}

    \question[] Ejercicio 3\droppoints
  1)   ¿Cuántas referencias a estos objetos
existen luego de que el código se ha ejecutado? ¿Es alguno de los objetos candidato a
ser eliminado por el garbage collector?

  \begin{verbatim}
String[] students = new String[10];
String studentName = "Peter Parker";
students[0] = studentName;
studentName = null;
...
  \end{verbatim}    
    \begin{solution}
      \textbf{studentName} hace referencia en principio al String 'Peter Parker'. Posteriormente \textbf{students[0]} referencia al mismo String. Al igualar \textbf{studentName} a \textbf{null}, el String 'Peter Parker' deja de ser referenciado por este. El array siempre está referenciado por la variable \textbf{students}. \\ 
      Dicho esto, ambos están siendo referenciados una vez, por lo que el garbage collector los deja tranquilos, al menos hasta que dejen de ser referenciados.
    \end{solution}

  2) ¿Cómo hace un programa para destruir  un objeto que ha creado?

  \begin{solution}
    En Java, un objeto se destruye cuando deja de ser referenciado. Por esta razón, cambiar todas las referencias al objeto en cuestión lo deja a la merced del garbage collector, que fríamente cumplirá con su tarea, arrasando con todos los objetos que nadie recuerda.
  \end{solution}

3) Dada la clase “ContenedorDeNumeros”, escribe un programa que
cree una instancia de la clase, inicialice sus dos variables miembro yluego muestre el valor de cada una de ellas.

\begin{solution}
  \begin{verbatim}
  NumberHolder nh = new NumberHolder();
  nh.aFloat = 2.0F;
  nh.anInt = 3;
  System.out.println ("nh.aFloat: " + nh.aFloat);
  System.out.println ("nh.anInt: " + nh.anInt);
  \end{verbatim}
\end{solution}
    
\end{questions}
\end{document}
