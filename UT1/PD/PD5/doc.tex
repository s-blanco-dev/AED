\documentclass[12pt,letterpaper, onecolumn]{exam}
\usepackage{amsmath}
\usepackage[T1]{fontenc}
\usepackage{amssymb}
\usepackage[a4paper, total={6.5in, 10in}]{geometry}
\usepackage[spanish]{babel}
% \chead{\hline} % Un-comment to draw line below header
\thispagestyle{empty}   %For removing header/footer from page 1

\begin{document}

\begingroup  
    \centering
    \LARGE Algoritmos y Estructuras de Datos\\
    \large Unidad Temática 1\\
    \large Práctico Individual 5\\[0.5em]
    \normalsize \today\\[0.5em]
    \normalsize Santiago Blanco Canaparro\par
    \normalsize Profesor: Sebastián Torres\par
    \normalsize Grupo I2M2\par
\endgroup
\rule{\textwidth}{0.4pt}
\pointsdroppedatright   %Self-explanatory
\printanswers
\renewcommand{\solutiontitle}{\noindent\textbf{Respuesta:}\enspace}   %Replace "Ans:" with starting keyword in solution box

\begin{questions}

  \question[] \textbf{Ejercicio 1}\droppoints
    
(1.) Escribe un ejemplo de uso de tal método, y asegúrate de comprender cómo funciona.
\vspace{-8mm}
    \begin{solution}
      \begin{verbatim}

public enum ProgrammingLanguages {
    C(10, 10),
    CPP(9, 9),
    GOLANG(8, 9),
    PYTHON(2, 3),
    JAVASCRIPT(1, 1),
    JAVA(5, 7),
    CSHARP(7, 7);

    private final int performance;
    private final int quality;

    ProgrammingLanguages(int performance, int quality) {
        this.performance = performance;
        this.quality = quality;
    }

    private int getPerformance() {return performance;}
    private int getQuality() {return quality;}

    int getTotal() {
        return getPerformance() + getQuality();
    }
}
      \end{verbatim}
          \end{solution}

(2.) Teniendo presente el programa que tu Equipo escribió
para contar vocales y
consonantes en una cierta frase, ¿cómo podrías escribirlo nuevamente utilizando tipos
enumerados?
    \begin{solution}
      \\
      Archivos adjuntos: \textbf{ContadorPalabras.java} y \textbf{Letra.java}
        \end{solution}
        
\noindent\rule{8cm}{0.4pt}
    
        \question[] \textbf{Ejercio 3: Strings}\\
       
    \vspace{-5mm}
    \begin{solution}
           \begin{verbatim}
public class StringDemo {
    public static void main(String[] args) {
        String palindrome = "bU;uB";
        int len = palindrome.length();
        char[] tempCharArray = new char[len];
        char[] charArray = new char[len];

        // put original string in an
        // array of chars
        for (int i = 0; i < len; i++) {
            tempCharArray[i] =
                    palindrome.charAt(i);
        }

        // reverse array of chars
        for (int j = 0; j < len; j++) {
            charArray[j] =
                    tempCharArray[len - 1 - j];
        }

        String reversePalindrome =
                new String(charArray);
        System.out.println(reversePalindrome);
    }
}
      \end{verbatim}
      El programa no normaliza las mayúsculas y minúsculas, toma la misma letra mínusculas como diferente a la misma en mayúsculas. 
      En cuanto a los símbolos de puntuación, los incluye como válidos.
    \end{solution}
\noindent\rule{8cm}{0.4pt}

\question[] \textbf{Ejercicio 4: Conversión de strings en números}\droppoints
  
    \begin{solution}
      El programa tetnía tres errores:
      \begin{verbatim}

      if (args.length == 3) {
      // debería recibir dos argumentos (length == 2) 

      float a = (Float.value (args[0])).floatValue();
      // debería ser 'valueOf' en vez de 'value'

      float b = (Float.valueOf(args[1])).float (); 
      // debería ser floatValue() en vez de 'float'
      \end{verbatim}

      Arreglados esos errores, la salida es para los parámetros 13.4 y 66.1 es:
      \begin{verbatim}
      a + b = 79.5
      a - b = -52.699997
      a * b = 885.7399
      a / b = 0.20272315
      a % b = 13.4
      \end{verbatim}

      Para un entero se debería modificar de la siguiente manera:
      \begin{verbatim}
                int a = Integer.parseInt(args[0]);
                int b = Integer.parseInt(args[1]);
      \end{verbatim}
         \end{solution}


\noindent\rule{8cm}{0.4pt}
\question[] \textbf{Ejercicio 5: Conversión de números en strings}\droppoints

\begin{solution}
  La salida:
  \begin{verbatim}
3 digits before decimal point.
2 digits after decimal point.
  \end{verbatim}

  \textbf{Double.toString(d)} convierte el valor de \textit{d} en String, mientras que la función \textbf{indexOf()} busca el índice en el ahora double convertido en String.
\end{solution}


\noindent\rule{8cm}{0.4pt}
\newpage
\question[] \textbf{Ejercicio 6: Métodos muy útiles de Strings.}\droppoints

\begin{solution}
  \textbf{archivo.txt} adjunto.
\end{solution}

\noindent\rule{8cm}{0.4pt}

\question[] \textbf{Ejercicio 7: StringBuilder}\droppoints
\begin{solution}
Los StringBuilder son Strings que pueden ser modificables, es decir, mutables. Son útiles para modificar la misma cadena sin necesedidad de crear nuevos Strings. \\
  \textbf{archivo2.txt} adjunto.
\end{solution}


\question[] \textbf{Ejercicio 8}\droppoints

¿Cuál es la capacidad inicial del siguiente stringbuilder?
\begin{verbatim}
StringBuilder sb = new StringBuilder("Able was I ere I saw
Elba.");
\end{verbatim}

\begin{solution}
  El String tiene un largo de 25 caracteres. De acuerdo con la lectura previa, el StringBuilder, cuando se le pasa por parámetro un String en el constructor, agrega 16 espacios además de la longitud del String: \\
  \textbf{Ergo: 25 + 16 = 41}
\end{solution}


\question[] \textbf{Ejercicio 9}\droppoints

\begin{solution} \\
  a) hannah.length() devuelve: 32 \\
  b) hannah.charAt(12) devuelve: 'e' \\
  c) hannah.charAt(15) devuelve 'b'
\end{solution}


\question[] \textbf{Ejercicio 10}\droppoints

\begin{solution} \\
  Devuelve 'car', siendo esta de 3 carácteres de largo. Se extraen las posiciones 9, 10 y 11
\end{solution}


\question[] \textbf{Ejercicio 11}\droppoints
\begin{solution}
\begin{verbatim}
1. eola
2. e2la
3. e 2la
4. e 2laste
5. e 2am laste
\end{verbatim}

\end{solution}
  \end{questions}
\end{document}
